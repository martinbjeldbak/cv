\documentclass[margin,line,a4paper]{resume}
 
\usepackage{polyglossia}
\setdefaultlanguage{danish}
\usepackage[none]{hyphenat}
\usepackage{graphicx,wrapfig}
\usepackage{url}
\usepackage{fontspec}
\usepackage{xltxtra}
\setmainfont{Minion Pro}
\usepackage[colorlinks=true, a4paper=true, pdfstartview=FitV,
linkcolor=blue, citecolor=blue, urlcolor=blue]{hyperref}
\usepackage[activate={true,nocompatibility},final,factor=1100,stretch=10,shrink=10,expansion=false,verbose=silent]{microtype}
\frenchspacing
 
\begin{document}
\raggedright
{\sc \Large Curriculum Vitae -- Martin Bjeldbak Madsen}
\begin{resume}
    \vspace{0.5cm}
    \begin{wrapfigure}{R}{0.6\textwidth}
         \vspace{-1cm}
        \begin{center}
        \reflectbox{\includegraphics[width=0.4\textwidth]{moi.jpg}}
        \end{center}
         \vspace{-2cm}
    \end{wrapfigure}

    \section{\mysidestyle Information}%\vspace{2mm}
    Martin Bjeldbak Madsen\\
    Født 9.\ juni 1992 (22 år)\\ 
    Lærkeparken 20\\
    6230 Rødekro\\
    Danmark\\
    Mobil: 26 27 36 60\\
    \href{http://martinbmadsen.dk}{martinbmadsen.dk}\\
    \href{mailto:martinbmadsen@gmail.com}{martinbmadsen@gmail.com}\\
    \href{http://dk.linkedin.com/in/martinbmadsen}{linkedin.com/in/martinbmadsen}
    \vspace{1cm}

    Jeg er født og opvokset i Danmark til jeg var 5 år, hvorefter vi havde et
    9-årigt langt ophold i udlandet både i England og USA. Så jeg er
    vant til at tale engelsk og dansk med de lokale, samt tilpasse mig
    til nye miljøer. Jeg bor selvstændigt og i fremtiden ser mig selv
    arbejde inden for teknologiindustrien, medicinalindustrien, eller
    bilindustrien.

    I øjeblikket læser jeg det første år af min kandidagrade i datalogi
    fra Aalborg Universitet på et udlandsophold på University of
    California, San Diego. Her går jeg på 7.\ semester og engagerer mig
    i nogle spændende projekter baseret på en dynamisk grundlag bygget
    op på problembaseret gruppearbejde! Vi har f.\ eks.\ designet op
    implementeret et programmeringssprog fra bunden af, der skal gøre
    det lettere, at hurtigt kunne skrive brætspil for derefter at spille
    dem direkte, enten mod computeren eller en anden spiller. Dette
    semester drejer sig om maskinintelligens, som interesserer mig
    meget.

    \section{\mysidestyle Uddannelse}
    Jeg har hvad der ligner en meget almen uddannelsesbaggrund. I
    fremtiden skal jeg til udlandet (USA) på det første år af min
    kandidat, hvorefter den færdiggøres på Aalborg Universitet.

    \textbf{Aalborg Universitet - Kandidat i Datalogi (MSC)} (2014 -
    2016) Er nu i gang med en kandidatgrad i datalogi.

    \textbf{University of California, San Diego - Computer Science}
    (2014 - 2015) Følger kurser på graduate (kandidat) niveau
    i forbindelse med et-årigt langt studieophold som led i min
    kandidatuddannelse fra Aalborg Universitet.

    \textbf{Aalborg Universitet - Bachelor i Datalogi (BSc)}
      (2011 - 2014) Afsluttet i juli 2014 med et vægtede gennemsnit på
      10.

    \textbf{EUC Nord - HTX} (2008 - 2011) Teknisk gymnasium i Hjørring.
      Naturvidenskabelig linje (Matematik A, Fysik A) samt Dansk A med
      Engelsk A og Elektronik A som valgfag.

    \textbf{Bagterpskolen} (2006 - 2008) Folkeskole i Hjørring.

    \textbf{J. R. Gerrits Middle School} (2003 - 2006) Elementary og
      Middle School i Kimberly, Wisconsin, USA.

\section{\mysidestyle Professionel\\erfaring}\vspace{1mm}
\begin{description}

  \item[2012 sept $\rightarrow$ 2014 aug] Studenterprogrammør hos Falck
    Healthcare a/s. Hovedsaglig webudvikling i Ruby on Rails på et internt
    system brugt dagligt af ca.\ 40 fysioterapeuter.

  \item[2010 sept $\rightarrow$ 2014 jun] Startede enkeltmandsvirksomheden
    \emph{Divambu}. Virksomheden har fokus på udvikling og hosting af
    hjemmesider samt konsultentservicer inden for IT og computere. På
    \url{divambu.dk} ses en portfolio af projekter jeg har gennemført for
    kunder. Hobbyvirksomhed med lav omsætning.

  \item[2009 feb $\rightarrow$ 2010 jun] Ungarbejder ved Fakta
    a/s. Jobfunktionerne bestod i at side ved kassen, fylde varer op,
    ordne dagligdags butiksarbejde, osv. Meget af jobbet bestod af at have
    kontakt med kunderne, forstå deres problem, og derefter løse det.
\end{description}

\section{\mysidestyle Andet aktivitet}\vspace{1mm}
\begin{description}
  \item[2012 juli $\rightarrow$ 2014 mar] Mentor, Projekt
    \href{http://www.urk.dk/solskinsunge/}{Solskinsunge} 2012. Frivillig
    arbejde, hvor vi arrangerer sociale arrangementer med socialt udsatte
    børn fra Tornhøjskolen i Aalborg Øst.
\end{description}
%Derudover har jeg været sponsoransvarlig for sommercampen Software
%Development Camp som UNF, Ungdommens Naturvidenskabelig Forening,
%afholdte i sommeren 2013. Her oprettede jeg kontakten med de
%forskellige virksomheder og organisationer og søgte sponsorat, så campen
%kunne afholdes.

%I januar 2013 blev jeg valgt ind i Studienævn
%for Datalogi på Aalborg Universitet
%(\href{http://www.sict.aau.dk/studienaevn-for-datalogi/}{link}),
%der dækker sig over uddannelserne Datalogi, Informatik,
%Informationsteknologi, Softwareingeniør, Softwarekonstruktion
%og tilhørende studieordninger. Her påtager vi os af
%dispensationsansøgninger, meritansøgninger, kvalitetssikring af
%uddannelsen osv. På grund af mit udlandsophold har jeg ikke længere
%mulighed at fortsætte i studienævnet.

Jeg har har haft dansk kørekort til bil i 4 år.

\section{\mysidestyle Kompetencer} \vspace{1mm}
Jeg kan hurtigt sætte mig ind i noget nyt, men der er nogle særlige
emner, der især interesserer mig:
\vspace{0.5cm}
\begin{description}
  \item[Styresystemer] Grundlæggende forståelse for de fleste
    Unix-lignende styresystemer: Ubuntu, Debian, CentOS, Arch Linux, Mac OS
    X.
  \item[Servere og databaser] Nginx. 
  \item[CMS-systemer] Jeg har opsat systemer med Wordpress, phpBB,
    Octopress.
  \item[Versionskontrolsystemer] Git, SVN.
  \item[Programmerings-, scriptings- og markupssprog] \LaTeX{} og
  \XeTeX{}, HTML, CSS, C, C$\sharp$, Java, JavaScript og jQuery, Ruby samt Ruby
  on Rails.
\end{description}

\section{\mysidestyle Sproglige kompetencer}
Mit modersmål er dansk, men næsten alt det jeg laver er på engelsk, både
i forbindelse med computere men også kommunikationen med internationale
venner, samt i en akademisk forstand på universitetet.

\begin{description}
  \item[Dansk] modersmål
  \item[Engelsk sprogligt] flydende
  \item[Engelsk skriftligt] flydende
\end{description}

\section{\mysidestyle Interesser}
Når jeg ikke sidder foran computeren, elsker jeg at nyde jeg en kop
te med en god bog i hånden. Der er heller ikke noget mere spændende
i livet end at tage ud i verden for at rejse rundt uden for trykke
Danmark. Heldigvis har jeg fundet nogle gode venner i udlandet, som jeg
engang imellem får fornøjelsen at besøge! Jeg er også stor tilhænger af
elektronisk musik som trance og house. Udover dette nyder jeg også at
lave lækkert og sundt mad, der kan få en til at savle!

\end{resume}
\end{document}
