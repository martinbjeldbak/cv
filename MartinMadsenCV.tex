% !TeX root = martinmadsenCV.tex
% !TEX program = xelatex

\documentclass{afriggeri-cv/friggeri-cv}

\defaultfontfeatures{Path=fonts/}
\usepackage{fontspec}
\usepackage{fontawesome}

\setmathfont{XITS Math}
\newfontfamily\bodyfont[]{LiberationSans-Regular}
\newfontfamily\thinfont[]{LiberationSans-Regular}
\newfontfamily\headingfont[]{LiberationSans-Bold}

\defaultfontfeatures{Mapping=tex-text}
\setmainfont[Mapping=tex-text, Color=textcolor,Path=fonts/]{LiberationSans-Regular}

\newcommand{\aau}{%
  Aalborg University (AAU), Denmark
}

\begin{document}
\header{martin}{madsen}
{software developer, \href{mailto:me@martinbjeldbak.com}{me@martinbjeldbak.com}}

\begin{aside}
  \section{about}
    melbourne
    australia
    \hfill
    29 years old
    \phone{+61 413 043 210}
    \website{martinbjeldbak.com}
    \github{martinbjeldbak}
    \linkedin{martinbjeldbak}
  \section{languages}
    bilingual Danish/English
    IELTS score: 8.5
  \section{programming}
    Ruby on Rails
    Golang
    Heroku, AWS (EC2, RDS, ECS, S3), SQL
    GitHub Actions
    \LaTeX, HTML, Git
    JavaScript, Python
    OS X, Unix
  \section{interests}
    continuous delivery
    machine learning
    devOps
    TDD, functional programming
    service oriented architecture
    web development
\end{aside}

\section{experience}

\begin{entrylist}
  \entry%
    {since 2020}
    {Software Engineer}
    {\href{https://www.anz.com.au}{ANZ Bank}, Melbourne, Australia}
    {Working on \href{https://www.anz.com.au/plus}{ANZ Plus} as software engineer embedded in an
     integration services team. I am responsible for implementing \& maintaining DevSecOps pipelines
     for feature engineers and developing microservices, mainly using GitHub Actions and Golang, respectively.}
  \entry%
    {2016--2020}
    {Lead Software Developer, Backend}
    {\href{https://netengine.com.au/}{NetEngine}, Brisbane, Australia}
    {Ruby on Rails software developer working with consultancy clients to build web-based systems across
      many different industries, from finance to flood mapping. I worked on setting up infrastructure
      in AWS to building GraphQL Rails APIs.
      I also partook in non-technical activities such as business planning, training new backend developers,
      etc. Clients included \href{https://www.apollocamper.com}{Apollo Motorhome Holidays}.}
  \entry%
    {2012--2014}
    {Student Software Developer}
    {\href{https://www.falck.com/services/healthcare}{Falck Healthcare a/s}, Aalborg, Denmark}
    {Development of Ruby on Rails-based system used daily by ca.\ 40 of Falck's physiotherapists to journal their patients' treatments and wellbeing.}
\end{entrylist}

\section{education}

\begin{entrylist}
  \entry%
    {2014--2016}
    {M.Sc.\ {\normalfont\ in Computer Science}}
    {\aau}
    {Majoring in Computer Science (120 ECTS)\\
    Specializing in Machine Intelligence}
  \entry%
    {2014--2015}
    {Studying abroad}
    {University of California, San Diego (UCSD), USA}
    {Fully accredited as part of the first year of my master's degree from Aalborg University}
  \entry%
    {2011--2014}
    {B.Sc.\ {\normalfont\ in Computer Science}}
    {\aau}
    {Majoring in Computer Science (180 ECTS)}
\end{entrylist}

\section{projects}
\begin{entrylist}
  \entry%
    {2015}
    {XQuery Processor}
    {UC San Diego, \github{martinbjeldbak/xquery-processor}}
    {Implemented a subset of the XQuery and XPath XML query languages in Java}
  \entry%
    {2016}
    {A Hierarchical Tree Distance Measure for Classification}
    {\href{https://martinbjeldbak.com/thesis.pdf}{thesis.pdf}}
    {Concluded my M.Sc.\ degree. Published in \href{https://www.scitepress.org/Papers/2017/61985/}{Proceedings of the 6th International Conference on Pattern Recognition Applications and Methods~-~Volume 1: ICPRAM 2017, pages 502--509}}
\end{entrylist}
\end{document}
